\section*{Introduction}
Precision oncology denotes treatment schemes in cancer in which medical decisions depend on the individual molecular status of a patient \cite{Garraway2013}. 
The most relevant molecular information is the patient’s genome, or, more precisely, the set of variations (mutations) each individual carries. 
Today a number of diagnosis and treatment options already depend on the (non-)existence of certain variations in a tumor \cite{Topalian2016}. 
When faced with the variant profile of a patient, clinicians critically depend on accurate, up-to-date, and detailed information regarding the clinical relevance of the present variations. Finding such information is highly laborious and time-consuming, often taking hours or even days \cite{Doig2017}. 
Clinicians use multiple search engines, such as PubMed, clinicaltrials.gov, or asco.org, and multiple databases, such as Cosmic \cite{Forbes2015}, DrugBank \cite{Wishart2008}, or Ensemble \cite{Varadi2015}. 
Especially important are specialized databases providing manually curated evidences for variation-therapy associations, such as OncoKB \cite{Chakravarty2017}, ClinVar \cite{Landrum2016}, or CIViC \cite{Griffith2017}. 
We report on VIST, a search engine specifically developed to aid clinicians in precision oncology in their search for clinically relevant information for a (set of) variations or mutated genes. 
As such it was developed to support the inner workings of a tumor board which goal is to determine the best possible cancer treatment and care plan for an individual patient.
As for any search engine, the core of VIST is its ranking function which, given a (set of) variation or a (set of) gene and a cancer entity, ranks those documents of its corpus highest which contain clinically relevant information. The main difficulty when developing a ranking function for such a novel and quickly emerging field are (a) the lack of gold standard data and (b) the complexity of the concept "clinical relevance", which encompasses information about gene-mutation-drug associations, frequencies of variations within populations, clinical trials, mode of action of drugs, molecular functions and pathways associated with a variation, reports on treatments of similar tumors, etc. When developing VIST we took two measures to cope with this complexity: (1) We use advanced information extraction to pre-filter documents based on the genes and variations they mention, and we (2) developed a document classifier using a silver-standard corpus of clinically related documents obtained from two different sources. VIST furthermore offers several metadata filters (journal, year of publication), highlights key phrases (i.e., the clinically most important sentences) and mentions of query entities when displaying documents, links out to external databases, and allows mixing of entity and classical keyword search. Furthermore, we performed a user study to obtain a set of <query, document, relevance> triples which allows for a systematic comparison to other search engines, showing that VIST’s ranking outperforms that of PubMed and a pure vector space model (VSM) in most cases. VIST was developed in close interaction with medical experts and is freely available at https://triage.informatik.huberlin.de:8080/

%\section*{Related work}
There are a number of search engines specialized for biomedical applications, but none that focuses specifically on precision oncology and clinical relevance. The most popular engine, PubMed, essentially search ranks results by the date of publication \cite{Fiorini2017}. Tools like GeneView \cite{Thomas2012}, PubTator \cite{Wei2013a} or SemeDa \cite{Kohler2003} pre-annotate documents in their index using various Named Entity Recognition (NER) tools to allow searching across spelling variations and synonyms. They also highlight recognized entities in matching documents. DigSee \cite{Kim2013} performs keyphrase detection for sentences describing the relationship between genes and diseases. DeepLife \cite{Ernst} also performs entity recognition and, in contrast to the previous tools which all consider only PubMed abstracts, also indexes certain web sites and social media content. RefMED \cite{Yu2009} facilitates search in PubMed by user relevance feedback. In contrast to VIST, none of these tools ranks according to a specific thematic focus of documents.

There are also a few search tools which are topically closer to VIST. The Cancer Hallmarks Analytics Tool (CHAT) \cite{Baker2017} classifies literature based on the predefined cancer hallmarks taxonomy. ASCOT \cite{Korkontzelos2012} searches texts in clinical trials, but has no cancer focus and does not apply advanced information extraction to find genes or variations. DGIdb \cite{Cotto2017} offers search over a database of text-mined clinically relevant drug-gene pairs; in contrast, VIST returns entire documents and has a much wider scope than just drug-gene pairs. A valuable recent contribution is the TREC Precision Medicine Track \cite{Roberts2017}, started in 2017 (and available again this year). INCORPORATE RELEVANT BIONLP/ACL PAPERS (\cite{W18-2313}, \cite{W18-2310}, \cite{seva2018}). 
